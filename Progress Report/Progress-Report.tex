% Options for packages loaded elsewhere
\PassOptionsToPackage{unicode}{hyperref}
\PassOptionsToPackage{hyphens}{url}
%
\documentclass[
]{article}
\usepackage{amsmath,amssymb}
\usepackage{iftex}
\ifPDFTeX
  \usepackage[T1]{fontenc}
  \usepackage[utf8]{inputenc}
  \usepackage{textcomp} % provide euro and other symbols
\else % if luatex or xetex
  \usepackage{unicode-math} % this also loads fontspec
  \defaultfontfeatures{Scale=MatchLowercase}
  \defaultfontfeatures[\rmfamily]{Ligatures=TeX,Scale=1}
\fi
\usepackage{lmodern}
\ifPDFTeX\else
  % xetex/luatex font selection
\fi
% Use upquote if available, for straight quotes in verbatim environments
\IfFileExists{upquote.sty}{\usepackage{upquote}}{}
\IfFileExists{microtype.sty}{% use microtype if available
  \usepackage[]{microtype}
  \UseMicrotypeSet[protrusion]{basicmath} % disable protrusion for tt fonts
}{}
\makeatletter
\@ifundefined{KOMAClassName}{% if non-KOMA class
  \IfFileExists{parskip.sty}{%
    \usepackage{parskip}
  }{% else
    \setlength{\parindent}{0pt}
    \setlength{\parskip}{6pt plus 2pt minus 1pt}}
}{% if KOMA class
  \KOMAoptions{parskip=half}}
\makeatother
\usepackage{xcolor}
\usepackage[margin=1in]{geometry}
\usepackage{longtable,booktabs,array}
\usepackage{calc} % for calculating minipage widths
% Correct order of tables after \paragraph or \subparagraph
\usepackage{etoolbox}
\makeatletter
\patchcmd\longtable{\par}{\if@noskipsec\mbox{}\fi\par}{}{}
\makeatother
% Allow footnotes in longtable head/foot
\IfFileExists{footnotehyper.sty}{\usepackage{footnotehyper}}{\usepackage{footnote}}
\makesavenoteenv{longtable}
\usepackage{graphicx}
\makeatletter
\def\maxwidth{\ifdim\Gin@nat@width>\linewidth\linewidth\else\Gin@nat@width\fi}
\def\maxheight{\ifdim\Gin@nat@height>\textheight\textheight\else\Gin@nat@height\fi}
\makeatother
% Scale images if necessary, so that they will not overflow the page
% margins by default, and it is still possible to overwrite the defaults
% using explicit options in \includegraphics[width, height, ...]{}
\setkeys{Gin}{width=\maxwidth,height=\maxheight,keepaspectratio}
% Set default figure placement to htbp
\makeatletter
\def\fps@figure{htbp}
\makeatother
\setlength{\emergencystretch}{3em} % prevent overfull lines
\providecommand{\tightlist}{%
  \setlength{\itemsep}{0pt}\setlength{\parskip}{0pt}}
\setcounter{secnumdepth}{-\maxdimen} % remove section numbering
\ifLuaTeX
  \usepackage{selnolig}  % disable illegal ligatures
\fi
\IfFileExists{bookmark.sty}{\usepackage{bookmark}}{\usepackage{hyperref}}
\IfFileExists{xurl.sty}{\usepackage{xurl}}{} % add URL line breaks if available
\urlstyle{same}
\hypersetup{
  hidelinks,
  pdfcreator={LaTeX via pandoc}}

\author{}
\date{\vspace{-2.5em}}

\begin{document}

\begin{longtable}[]{@{}l@{}}
\toprule\noalign{}
\endhead
\bottomrule\noalign{}
\endlastfoot
header-includes: \\
- \\
- \\
- \\
- \\
- \pagestyle{fancy} \\
- \fancypagestyle{plain}{\pagestyle{fancy}} \\
- \headheight 35pt \\
- \fancyhead[LE,LO]{Abowath, Blakely, \\ Garrison, Loui, Peir} \\
-
\fancyhead[CO,CE]{\textbf{\Large Forecasting Bikesharing Usage \\ Team 10 - Progress Report}} \\
- \fancyhead[RE,RO]{MGT 6203 \\ Fall 2023} \\
- \fancyfoot[RE,RO]{\small \thepage} \\
- \fancyfoot[CE,CO]{} \\
- \headsep 1.5em \\
output: pdf\_document \\
fontsize: 11pt \\
\end{longtable}

\hypertarget{forecasting-bikesharing-usage-for-dcs-capital-bikshare-system}{%
\section{Forecasting Bikesharing Usage for DC's Capital Bikshare
System}\label{forecasting-bikesharing-usage-for-dcs-capital-bikshare-system}}

\hypertarget{table-of-contents}{%
\section{Table of Contents}\label{table-of-contents}}

I. Introduction

\begin{enumerate}
\def\labelenumi{\Roman{enumi}.}
\setcounter{enumi}{1}
\item
  Current Status of the Project
\item
  Ongoing Work
\end{enumerate}

V. Literature Review Summary

\begin{enumerate}
\def\labelenumi{\Roman{enumi}.}
\setcounter{enumi}{5}
\tightlist
\item
  Works Cited
\end{enumerate}

\hypertarget{introduction}{%
\section{Introduction}\label{introduction}}

Bikesharing systems are an increasingly popular solution in major urban
areas to increase trips taken by bike, which can help people get around
without cars, thus improving the lives of both users, as well as
non-users, as each bike trip potentially represents a trip that would
otherwise have required a car. We hope to use data from the DC Capital
Bikeshare in 2011 and 2012 to predict bikeshare usage system-wide.

The purpose of this analysis is to determine variables/factors that help
estimate bikeshare useage and develop a model that predicts bikeshare
useage based on certain predictor variables.

\hypertarget{current-status-of-the-project}{%
\section{Current Status of the
Project}\label{current-status-of-the-project}}

Currently, the dataset has undergone exploratory data analysis and data
cleaning. We've identified predictors that may impact a potential
model's goodness of fit and converted some variables into categorical
variables.

\hypertarget{data-cleaning-and-preprocessing}{%
\subsection{Data Cleaning and
Preprocessing}\label{data-cleaning-and-preprocessing}}

The dataset has required minimal cleaning. We had to convert several
variables into factor variables (season, holiday, weekday, workingday,
weather). Additionally we noted that the data key for our dataset
mislabeled the season factor variable, which was trivial to correct.
Fortunately, there was no missing data.

\hypertarget{eda-visualization}{%
\subsection{EDA / Visualization}\label{eda-visualization}}

\begin{center}\includegraphics[width=0.33\linewidth]{Progress-Report_files/figure-latex/hist-1} \end{center}

\begin{center}\includegraphics[width=0.33\linewidth]{Progress-Report_files/figure-latex/hist-2} \end{center}

\begin{center}\includegraphics[width=0.33\linewidth]{Progress-Report_files/figure-latex/hist-3} \end{center}

\begin{center}\includegraphics[width=0.33\linewidth]{Progress-Report_files/figure-latex/hist-4} \end{center}

Not all of our predictors or response variables are distributed
normally. In particular, humidity exhibits leftward skew, and windspeed
exhibits rightward skew.

\begin{center}\includegraphics[width=0.75\linewidth]{Progress-Report_files/figure-latex/usage_hist-1} \end{center}

\begin{center}\includegraphics[width=0.75\linewidth]{Progress-Report_files/figure-latex/usage_hist-2} \end{center}

Casual users exhibit rightward skew, while registered users are not too
far from the normal distribution.

Preliminary EDA reveals strong seasonality - overall usage on a daily
basis is much higher in spring/summer than fall and much more than
winter.

We wanted to look at the overall bike usage as it relates to
seasonality. We had the hypothesis that we would have more users in the
summer than in the winter. Looking at the plot below, we can see that
not only do we have the max bike usage during the summer, but we also
seem to have an increase to bike usage as the service becomes more
mature.

\begin{center}\includegraphics[width=0.75\linewidth]{Progress-Report_files/figure-latex/seasonality_daily_usage-1} \end{center}

The next thing we wanted to take a look at is see whether the day of the
week had a large impact on the total bike usage. This is important to
see whether a specific day will have the majority of usage or if they
are relatively evenly spread out. Ideally we want to see the data spread
out. This would allow us to increase on the capitalization of renting
our bikes every day.

\begin{center}\includegraphics[width=0.75\linewidth]{Progress-Report_files/figure-latex/weekday_usage-1} \end{center}

Next, let's verify these findings by conducting an ANOVA and pairwise
analysis.

\begin{verbatim}
##                Df    Sum Sq  Mean Sq F value Pr(>F)    
## df_h$season     3  37729358 12576453   409.2 <2e-16 ***
## Residuals   17375 534032233    30736                   
## ---
## Signif. codes:  0 '***' 0.001 '**' 0.01 '*' 0.05 '.' 0.1 ' ' 1
\end{verbatim}

The p-value of the F-statistic for season is very small and
statistically significant.At least one group mean is different from the
rest.

Next, we compare the means of each pair of seasons.

\begin{verbatim}
##   Tukey multiple comparisons of means
##     95% family-wise confidence level
## 
## Fit: aov(formula = df_h$total_count ~ df_h$season, data = df_h)
## 
## $`df_h$season`
##                     diff       lwr         upr     p adj
## spring-winter  97.229500  87.54202 106.9169764 0.0000000
## summer-winter 124.901668 115.26026 134.5430748 0.0000000
## fall-winter    87.754288  77.96798  97.5405970 0.0000000
## summer-spring  27.672168  18.12517  37.2191613 0.0000000
## fall-spring    -9.475213 -19.16852   0.2180949 0.0581801
## fall-summer   -37.147380 -46.79465 -27.5001142 0.0000000
\end{verbatim}

From the Tukey method, all the pairs are statistically different except
the fall-spring pair. This confirms the visual in the box plot, where we
see similar distributions in fall and spring.

We can see by the above given boxplot, that the data appears to be
evenly spread out between each weekday and across all the seasons. This
shows that in general each day will yield approximately the same number
of riders, which will help increase our profit. The question though is
do we see a difference between casual and registered users.

\begin{center}\includegraphics{Progress-Report_files/figure-latex/casual_vs_registered-1} \end{center}

We can see above that the majority of the \texttt{casual} user's usage
is on the weekend, and the registered user uses them consistently
throughout the week. This indicates that we might need to look at
different models depending on the user type. We might see a little more
non-linearity with casual users than with the registered users.

\hypertarget{preliminary-time-series-forecasting}{%
\subsection{Preliminary Time Series
Forecasting}\label{preliminary-time-series-forecasting}}

We have started with a simple time series forecasting model using the
day wise usage data. We have used ARIMA to forecast for on the last 51
days of the year. We understand from the data analysis that we have
conducted above that seasonal effects do have an effect on the usage.

\begin{verbatim}
##                 ME     RMSE     MAE       MPE     MAPE
## Test set -915.7404 1846.462 1295.75 -70.53495 76.82173
\end{verbatim}

\begin{center}\includegraphics{Progress-Report_files/figure-latex/Premilinary Time series forecasting day wise-1} \end{center}

\hypertarget{ongoing-work}{%
\section{Ongoing Work}\label{ongoing-work}}

Anticipated challenges:

\begin{itemize}
\item
  Distinguishing between the influences of weather and the influence of
  the season, particularly on casual usage. DC has many visitors in
  Spring and Summer who would show up under the casual response
  variable.
\item
  Accounting for the overall increase in usage over the two years
  spanning our dataset. The dataset was collected towards the beginning
  of the Capital Bikeshare program, so findings we draw from the overall
  increasing usage would not necessarily translate directly to mature
  systems.
\end{itemize}

\hypertarget{future-modeling}{%
\subsection{Future Modeling}\label{future-modeling}}

The initial results of a time series forecasting model on a day level
depicts that our accuracy is pretty low and especially it is unable to
gauge the seasonal effects in the data especially the sudden drop of
usage in winters. In order to improve our models we will now explore
different modeling techniques such as Decision tree/Ensemble trees and
use train and test data set that has all the components of seasons and
other time level factors. In addition to this, we will also try to
improve on the time series model by either exploring the hourly patters
and forecasting for usage for a particular hours in a day. Lastly, we
will consider getting more years of data which will allow us to capture
the seasonal and time based patterns of usage in the model.

Additionally, as mentioned above, the two different response variables
\texttt{reigstered} and \texttt{casual} exhibit significant differences
and likely warrant building two separate models. Due to the correlation
of our temperature variables with seasons (logically) we may needs to
consider variable selection techniques to reduce the number of factors
down.

\hypertarget{literature-review-summary}{%
\section{Literature Review Summary}\label{literature-review-summary}}

Existing literature around bikeshare usage generally emphasizes the
following:

\begin{itemize}
\tightlist
\item
  Time of day is typically the most important predictor, but different
  days of the week have different trends based on time of day
\item
  Specifically, usage is often bimodal on weekdays reflecting commuter
  patterns
\item
  Usage is not bimodal on weekends, typically with the highest value in
  mid-afternoon.
\item
  Usage increases as temperature increases, then starts to decrease as
  temperatures go into the 90s (Fahrenheit), which can be too hot
\item
  Precipitation of any amount discourages cycling
\item
  High humidity has a negative effect on cycling
\item
  High winds can have a negative effect on cycling
\item
  Usage is often higher in spring and summer, and lowest in winter
\end{itemize}

\hypertarget{works-cited}{%
\section{Works Cited}\label{works-cited}}

Bean, R., Pojani, D., \& Corcoran, J. (2021). How does weather affect
bikeshare use? A comparative analysis of forty cities across climate
zones. \emph{Journal of Transport Geography}, 95.
\url{https://doi.org/10.1016/j.jtrangeo.2021.103155}.

Eren, E., \& Uz, V. E. (2020). A review on bike-sharing: The factors
affecting bike-sharing demand. \emph{Sustainable Cities and Society},
54. \url{https://doi.org/10.1016/j.scs.2019.101882}

Ashgar, H. I., Elhenawy, M., \& Rakha, H. A. (2019). Modeling bike
counts in a bike-sharing system considering the effect of weather
conditions. \emph{Case Studies on Transport Policy}, 7(2), 261-268.
\url{https://doi.org/10.1016/j.cstp.2019.02.011}

\end{document}
